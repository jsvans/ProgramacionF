\documentclass[12pt]{article}
\usepackage[spanish]{babel}
\selectlanguage{spanish}
\usepackage[utf8]{inputenc}
\title{Tutorial breve de los comandos de Bash}
\author{Linus Torsvald}
\date{28 de Enero de 2015}
\begin{document}

\maketitle
\begin{itemize}
\item {\tt pwd} ( Presenta el directorio en uso. Ej. pwd) 
\item {\tt Ls} (Enlista los contenidos de un directorio. Ej. Ls -l)
\item {\tt Cd} (Cambia de directorio. Ej. Cd AAA)
\item {\tt File} (Obtiene información sobre qué tipo de archivo o drectorio es. Ej. file AAA/file1.txt) 
\item {\tt ls –a} (Enlista los contenidos de un directorio, incluyendo archives ocultos. ls –a AAA)
\item {\tt man} (Abre el manual para un commando específico. Ej.  man cd) 
\item {\tt man -k } (Busca una letra o palabra en todos los manuals existentes. Ej. man ls)
\item {\tt Mkdir} (Crea un directorio. Ej. mkdir BBB)
\item {\tt Rmdir} (Elimina un directorio. Ej rmdir BBB/bar) 
\item {\tt Touch} (Crea un archivo en blanco. Ej. touch CCC)
\item {\tt Cp} (Copia un archivo o directorio. Ej. cp CCC)
\item {\tt Mv} (Mueve un archivo o directorio. Ej. mv CCC/BBB) 
\item {\tt Rm} (Elimina un archivo. Ej. rm CCC)
\item {\tt Vi} (Edita un archivo. Ej. vi BBB)
\item {\tt Cat} (Ver un archivo. Ej. cat BBB) 
\item {\tt less} (Ver el contenido de archivos)
\item {\tt Chmod} (Cambia los permisos de un archive o directorio.)
\item {\tt ls -ld} (Inspecciona los permisos de un directorio específico.)
\end{itemize}


\begin{enumerate}
\item {\tt pwd} ( Presenta el directorio en uso. Ej. pwd) 
\item {\tt Ls} (Enlista los contenidos de un directorio. Ej. Ls -l)
\item {\tt Cd} (Cambia de directorio. Ej. Cd AAA)
\item {\tt File} (Obtiene información sobre qué tipo de archivo o drectorio es. Ej. file AAA/file1.txt) 
\item {\tt ls –a} (Enlista los contenidos de un directorio, incluyendo archives ocultos. ls –a AAA)
\item {\tt man} (Abre el manual para un commando específico. Ej.  man cd) 
\item {\tt man -k } (Busca una letra o palabra en todos los manuals existentes. Ej. man ls)
\item {\tt Mkdir} (Crea un directorio. Ej. mkdir BBB)
\item {\tt Rmdir} (Elimina un directorio. Ej rmdir BBB/bar) 
\item {\tt Touch} (Crea un archivo en blanco. Ej. touch CCC)
\item {\tt Cp} (Copia un archivo o directorio. Ej. cp CCC)
\item {\tt Mv} (Mueve un archivo o directorio. Ej. mv CCC/BBB) 
\item {\tt Rm} (Elimina un archivo. Ej. rm CCC)
\item {\tt Vi} (Edita un archivo. Ej. vi BBB)
\item {\tt Cat} (Ver un archivo. Ej. cat BBB) 
\item {\tt less} (Ver el contenido de archivos)
\item {\tt Chmod} (Cambia los permisos de un archive o directorio.)
\item {\tt ls -ld} (Inspecciona los permisos de un directorio específico.)
\end{enumerate}
\begin{tabular}{|c|54|72|}
Comando & Descripción & Ejemplo \\ \hline
pwd & Presenta el directorio en uso. Ej. pwd \\ \hline 
Ls & Enlista los contenidos de un directorio. Ej. Ls -l \\ \hline 
Cd & Cambia de directorio. Ej. Cd AAA \\ \hline 
File & Obtiene información sobre qué tipo de archivo o drectorio es. Ej. file AAA/file1.txt \\ \hline 
ls –a & Enlista los contenidos de un directorio, incluyendo archives ocultos. ls –a AAA \\ \hline 
man<comando> & Abre el manual para un commando específico. Ej.  man cd \\ \hline 
man -k <termino a buscar> & Busca una letra o palabra en todos los manuals existentes. Ej. man ls \\ \hline 
Mkdir & Crea un directorio. Ej. mkdir BBB \\ \hline 
Rmdir & Elimina un directorio. Ej rmdir BBB/bar \\ \hline 
Touch & Crea un archivo en blanco. Ej. touch CCC \\ \hline 
Cp & Copia un archivo o directorio. Ej. cp CCC \\ \hline 
Mv & Mueve un archivo o directorio. Ej. mv CCC/BBB \\ \hline 
Rm & Elimina un archivo. Ej. rm CCC \\ \hline 
Vi & Edita un archivo. Ej. vi BBB \\ \hline 
Cat & Ver un archivo. Ej. cat BBB) \\ \hline 
less & Ver el contenido de archivos \\ \hline 
Chmod & Cambia los permisos de un archive o directorio. \\ \hline 
ls -ld & Inspecciona los permisos de un directorio específico. \\ \hline 
\end{tabular}
\end{document}